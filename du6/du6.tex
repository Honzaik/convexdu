\documentclass[12pt, a4paper]{article}
\usepackage[margin=1in]{geometry}
\usepackage[utf8x]{inputenc}
\usepackage{indentfirst} %indentace prvního odstavce
\usepackage{mathtools}
\usepackage{amsfonts}
\usepackage{amsmath}
\usepackage{amssymb}
\usepackage{graphicx}
\usepackage{enumitem}
\usepackage{subfig}
\usepackage{float}
\usepackage[czech]{babel}
\usepackage{mathdots}
\usepackage{slashbox}

\begin{document}

\section{}
Z definice \uv{feasible solution} víme, že pro každé $x,y \in X$ platí $\forall i: f_i(x) \leq 0$, kde $f_i$ jsou konvexní funkce z definice (P). Dále také platí $Ax=Ay=b$, kde $A,b$ je matice a vektor z definice (P). Zvolme $\theta \in [0,1]$ a dokážeme, že $\theta x + (1-\theta)y$ splňuje tyto rovnosti a nerovnosti, takže je také součástí $X$, takže $X$ je konvexní.
\begin{gather*}
\forall i: f_i(\theta x + (1-\theta)y) \stackrel{f_i \text{ konvex.}}{\leq} \theta f_i(x) + (1-\theta)f_i(y) \stackrel{f_i(x),f_i(y)\leq 0}{\leq} \theta 0 + (1-\theta)0 = 0\\
A(\theta x + (1-\theta)y)  \stackrel{\text{distrub. nasobeni matic}}{=} \theta Ax + (1-\theta) Ay \stackrel{Ax=Ay=b}{=} \theta b + (1-\theta)b = b
\end{gather*}
Z toho plyne $\theta x + (1-\theta)y \in X$. $X$ je tedy konvexní množina.

\section{}
Podmínku o semidefinitnosti matic můžeme zjednodušit (dosazením a sečteným) na to, že matice $H$ musí být negativně semidefinitní neboli $-H$ pozitivně semidefinitní:\\
\begin{minipage}{.5\linewidth}
\begin{gather*}
H = \begin{pmatrix}
-x_1+2x_2+100 & 2x_1+56\\
2x_1+56 & x_1 + x_2 + 100
\end{pmatrix}, H \preceq 0 \iff
\end{gather*}
\end{minipage}
\begin{minipage}{.5\linewidth}
\begin{equation}
x_1-2x_2-100 \geq 0
\end{equation}
\begin{equation}
-x_1-x_2-100 \geq 0
\end{equation}
\begin{equation}
det(-H) \geq 0
\end{equation}
\end{minipage}

Víme, že matice $2 \times 2$ je pozitivně semidefinitní právě tehdy, když prvky na diagonále jsou nezáporné a determinant je nezáporný, což přesně vyjadřují nerovnice výše.
\[
det(-H) = 2 x_2^2 + x_1x_2 +300x_2 -5x_1^2 -224x_1 +6864
\]

V programu $Q$ budeme mít celkem 6 proměnných $y_1, \dots, y_6$. Proměnné budou mít tento význam: $x_1 = x_1^+ - x_1^- = y_1 - y_2$, analogicky $x_2 = y_3 - y_4, x_3 = y_5 - y_6$, kde $y_i \geq 0$. Matice $Y$ bude typu $9 \times 9$, kde na diagonále budou proměnné $y_1,\dots, y_6$ a zbylá 3 místa na diagonále budou odpovídat rovnicím $(1),(2),(3)$.
\begin{gather*}
x_1-2x_2-100 \geq 0 \iff (y_1-y_2) - 2(y_3-y_4)-100 \geq 0\\
-x_1-x_2-100 \geq 0 \iff -(y_1-y_2) -(y_3-y_4)-100 \geq 0\\
det(-H) \geq 0\\
\iff \\
2(y_3-y_4)^2 + (y_1-y_2)(y_3-y_4) + 300(y_3-y_4) -5(y_1-y_2)^2 - 224(y_1-y_2)+6864 \geq 0\\
\implies \\
Y = diag(y_1, y_2, y_3, y_4, y_5, y_6, (y_1-y_2) - 2(y_3-y_4)-100,  -(y_1-y_2) -(y_3-y_4)-100, \\
2(y_3-y_4)^2 + (y_1-y_2)(y_3-y_4) + 300(y_3-y_4) -5(y_1-y_2)^2 - 224(y_1-y_2)+6864)
\end{gather*}
Matice $Y$ je diagonální, takže bude pozitivně semidefinitní právě tehdy, když všechny prvky na diagonále budou nezáporné, což nastane právě tehdy, když budou splněny dané nerovnice.
Maticí $A$ potřebujeme vyjádřit jedinou rovnost v $P$, toho docílíme, pokud
\begin{gather*}
A = diag(1,-1,1,-1,1,-1,0,0,0), b=1\\
\iff\\
(y_1-y_2)+(y_3-y_4)+(y_5-y_6) = 1
\end{gather*}
Matici $C$ definujeme také jako diagonální, aby platila ekvivalentní min. podmínka, takže
\[
C = diag(5,-5,-1,1,0,0,0,0,0)
\]
Díky tomu, že matice jsou diagonální je zřejmé, že podmínky jsou ekvivalentní.

\end{document}
