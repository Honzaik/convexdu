\documentclass[12pt, a4paper]{article}
\usepackage[margin=1in]{geometry}
\usepackage[utf8x]{inputenc}
\usepackage{indentfirst} %indentace prvního odstavce
\usepackage{mathtools}
\usepackage{amsfonts}
\usepackage{amsmath}
\usepackage{amssymb}
\usepackage{graphicx}
\usepackage{enumitem}
\usepackage{subfig}
\usepackage{float}
\usepackage[czech]{babel}
\usepackage{mathdots}
\usepackage{slashbox}

\begin{document}

\section{}
s

\section{}
m

\section{}
Z Farskasova lemma plyne, že pokud dokážeme, že existuje $y \in \mathbf{R}^3: A^T y \geq 0 \land b^T y < 0$, kde $b^T = (3, -2, 0)$ a 
\[
A = \begin{pmatrix}
0 & 0 & 2 & -1 & -1 & 0 & 0\\
1 & 1 & -1 & 1 & 0 & -1 & 0 \\
1 & 1 & 1 & 0 & 0 & 0 & 1
\end{pmatrix}
\]
($A$ reprezentuje zadanou soustavu), tak daná soustava nemá řešení. Stačí zvolit\\
$y^T = (-1, 0, 2)$:
\begin{gather*}
-1\begin{pmatrix}
0\\
0\\
2\\
-1\\
-1\\
0\\
0
\end{pmatrix} + 
2\begin{pmatrix}
1\\
1\\
1\\
0\\
0\\
0\\
1
\end{pmatrix} = 
\begin{pmatrix}
2\\
2\\
0\\
1\\
1\\
0\\
2
\end{pmatrix} \geq 0, \begin{pmatrix}3 & -2 & 0\end{pmatrix} \begin{pmatrix}
-1\\
0\\
2
\end{pmatrix} = -3 < 0
\end{gather*}

\section{}
dd

\end{document}
