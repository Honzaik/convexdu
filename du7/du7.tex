\documentclass[12pt, a4paper]{article}
\usepackage[margin=1in]{geometry}
\usepackage[utf8x]{inputenc}
\usepackage{indentfirst} %indentace prvního odstavce
\usepackage{mathtools}
\usepackage{amsfonts}
\usepackage{amsmath}
\usepackage{amssymb}
\usepackage{graphicx}
\usepackage{enumitem}
\usepackage{subfig}
\usepackage{float}
\usepackage[czech]{babel}
\usepackage{mathdots}
\usepackage{slashbox}

\begin{document}

\section{}
Dokážeme nejdříve implikaci "$K$ obsahuje přímku $\implies$ $K^*$ neobsahuje $n$-tici LN vektorů":

$K$ obsahuje přímku neboli existuje nenulový vektor $v \in K$ tž. $\forall \lambda \in \mathbb{R}: \lambda v \in K$, speciálně $-v,v \in K$. Mějme libovolnou $n$-tici vektorů $y_1,\dots,y_n \in K^*$. Z definice $K^*$ platí, že $\forall i: v^T y_i \geq 0 \land -(v^Ty_i) \geq 0 \implies v^Ty_i = 0$. Uvažujme matici
\[
A = \begin{pmatrix}
y_1^T\\
y_2^T\\
\vdots\\
y_n^T
\end{pmatrix} \implies Av = \begin{pmatrix}
v^Ty_1\\
v^Ty_2\\
\vdots\\
v^Ty_n
\end{pmatrix}
\]
Z lineární algebry víme, že matice $A$ je singulární $\iff$ existuje nenulový vektor $x$ tž. $Ax=0$. Tento vektor existuje a je to právě $v$. Dále víme, že $A$ je singulární $\iff$ řádky $A$ jsou lineárně závislé. Takže $y_1,\dots,y_n$ je lineárně závislá posloupnost.

Nyní dokážeme implikaci "$K$ neobsahuje $n$-tici LN vektorů $\implies K^*$ obsahuje přímku":

Pokud $K$ neobsahuje $n$-tici LN vektorů, tak $K$ je obsažen v podprostoru $R^n$ menší dimenze než $n$ (lineární obal $K$). Z toho plyne, že existuje vektor $v \in \mathbb{R}^n$ tž. je kolmý na daný podprostor (Gram-Schmidt), speciálně $\forall y \in K: v^T y = 0$. Z definice plyne $v \in K^*$ a také jeho libovolný násobek $cv$, kde $c \in R$, protože $\forall y \in K: (cv)^Ty = c(v^Ty) = c(0) = 0$.

Druhá implikace ze zadání ($K^*$ neobsahuje $n$-tici LN vektorů $\implies K$ obsahuje přímku") plyne z přechozích dvou:

Předpokládáme, že $K$ je uzavřená množina neboli $\overline{K} = K$ a díky faktu ze zadání tedy platí $(K^*)^* = K$. V předchozí implikaci tedy dosadíme místo $K$ $K^*$ a máme dokázanou zbývající implikaci.

\section{}
m

\section{}
Z Farskasova lemma plyne, že pokud dokážeme, že existuje $y \in \mathbf{R}^3: A^T y \geq 0 \land b^T y < 0$, kde $b^T = (3, -2, 0)$ a 
\[
A = \begin{pmatrix}
0 & 0 & 2 & -1 & -1 & 0 & 0\\
1 & 1 & -1 & 1 & 0 & -1 & 0 \\
1 & 1 & 1 & 0 & 0 & 0 & 1
\end{pmatrix}
\]
($A$ reprezentuje zadanou soustavu), tak daná soustava nemá řešení. Stačí zvolit\\
$y^T = (-1, 0, 2)$:
\begin{gather*}
-1\begin{pmatrix}
0\\
0\\
2\\
-1\\
-1\\
0\\
0
\end{pmatrix} + 
2\begin{pmatrix}
1\\
1\\
1\\
0\\
0\\
0\\
1
\end{pmatrix} = 
\begin{pmatrix}
2\\
2\\
0\\
1\\
1\\
0\\
2
\end{pmatrix} \geq 0, \begin{pmatrix}3 & -2 & 0\end{pmatrix} \begin{pmatrix}
-1\\
0\\
2
\end{pmatrix} = -3 < 0
\end{gather*}

\section{}
Dokážeme 2 inkluze: $int(S^n_+) \subseteq S^n_{++}$ a $S^n_{++} \subseteq int(S^n_+)$. Budeme uvažovat spektrální normu. První inkluze:

Zvolme $A \in int(S^n_+)$. Z definice tedy  $\exists \epsilon > 0, \forall X \in S^n, \Vert A-X \Vert < \epsilon \implies X \in S^n_+$. Zvolme $0 < \delta < \epsilon$, položme $X \coloneqq  A - \delta I_n$. Poté zřejmě $\Vert A - X \Vert = \Vert \delta I_n \Vert = \delta < \epsilon$. Takže $A-\delta I_n \in S^n_+$. Vlastní čísla matice $A-\delta I_n$ jsou právě $\lambda_i - \delta$, kde $\lambda_i$ je vlastní číslo $A$. To platí z definice: nechť $x$ vlastní vektor $A$ příslušný vlastnímu číslu $\lambda_i$ $\iff x$ je vlastní vektor $A-\delta I_n$ příslušný $\lambda_i - \delta$, protože:
\begin{gather*}
\implies: (A-\delta I_n)x = Ax - \delta I_n x = \lambda_i x - \delta x = (\lambda_i -\delta) x\\
\impliedby: Ax - \delta x = (A-\delta I_n)x = (\lambda_i - \delta)x = \lambda_i x - \delta x \implies Ax = \lambda_i x
\end{gather*}

$A-\delta I_n \in S^n_+ \implies \lambda_i-\delta \geq 0$, takže pro vlastní čísla $A$ platí $\lambda_i \geq \delta > 0$. Takže $A \in S^n_{++}$.

Nyní druhá inkluze:

Zvolme $A \in S^n_{++}$. Položme $\lambda$ jako nejmenší vlastní číslo $A$. $A$ je pozitivně definitní, takže $\lambda > 0$. Uvažujme kouli $S \coloneqq \{ X \in S^n: \Vert A-X \Vert < \frac{\lambda}{2}\}$. Dokážeme, že $S \subseteq S^n_+$ a tím bude platit druhá inkluze. Zvolme $X \in S$, z definice spektrální normy tedy plyne, že $\forall x \in R^n, \Vert x \Vert = 1: $
\end{document}
