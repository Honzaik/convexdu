\documentclass[12pt, a4paper]{article}
\usepackage[margin=1in]{geometry}
\usepackage[utf8x]{inputenc}
\usepackage{indentfirst} %indentace prvního odstavce
\usepackage{mathtools}
\usepackage{amsfonts}
\usepackage{amsmath}
\usepackage{amssymb}
\usepackage{graphicx}
\usepackage{enumitem}
\usepackage{subfig}
\usepackage{float}
\usepackage[czech]{babel}
\usepackage{mathdots}
\usepackage{slashbox}

\begin{document}

\section{}
Funkce $f$ je definována na množině $S = \{(x,y)^T \in \mathbb{R}^2: 3x+7y>10\}$ (kvůli logaritmu), což je konvexní množina (polorovina). Konvexitu funkce $f$ dokážeme, tak, že postupně dokážeme konvexitu 3 funkcí, ze kterých je $f$ složena ($f = f_1 + f_2 + f_3$). Víme, že součet konvexních funkcí je konvexní funkce, takže to stačí.\\
$f_1(x,y) = -57log(3x+7y-10)$: $-57log(x)$ je konvexní a neklesající, $g(x,y) = 3x+7y-10$ je konkávní (i konvexní) funkce (hesián je nulová matice). Z přednášky víme, že složením těchto funkcí dostaneme konvexní funkci.\\
$f_2(x,y) = 5x^2-xy+y^2$. Hesián je $\begin{pmatrix}
5 & -1\\
-1 && 1
\end{pmatrix}$
což je pozitivně semidefinitní matice. Takže $f_2$ je konvexní.
$f_3(x,y) = \text{ max. vl. číslo } \begin{pmatrix}
x & y\\
y & x+y
\end{pmatrix}$. Charakteristický polynom této matice je $p(\lambda)=\lambda^2 + \lambda(-2x-y) + (x^2-y^2)$. Tato kvadratická rovnice má 2 řešení neboli nám dává 2 funkce ($g_1,g_2$)v proměnných $x,y$: $g_1(x,y)=x-\frac{\sqrt{5}}{2}y-\frac{y}{2}, g_2(x,y)=x+\frac{\sqrt{5}}{2}y-\frac{y}{2}$. Tyto funkce jsou konvexní, jelikož jejich hesián je nulová matice. Funkci $f_3$ můžeme tedy vyjádřit takto $f_3(x,y)=max\{g_1(x,y),g_2(x,y)\}$. Z přednášky víme, že $f_3$ je konvexní, pokud $g_1,g_2$ jsou konvexní, což jsme dokázali.\\
Dokázali jsme tedy, že $f$ je konvexní funkce.

\section{}
$f(x,y,z) = g_1(x)g_2(y)g_3(z)$. Pro kladnost $f$ na $\mathbb{R}^3_{++}$ stačí dokázat kladnost všech $g_i$ na $\mathbb{R}_{++}$. První derivace funkcí $g_i$ jsou tvaru $e^{-x}$. Tato funkce je zřejmě kladná $\forall x \in \mathbb{R}$. Takže $g_i$ jsou ryze rostoucí na svém definičním oboru. Pro všechny $g_i$ platí, že v bodě $0$ mají hodnotu $1-1=0$. Z toho, že jsou $g_i$ ryze rostoucí můžeme tedy usoudit, že $\forall x \in \mathbb{R}_{++}: g_i(x)>0 \implies \forall (x,y,z) \in \mathbb{R}^3_{++}: f(x,y,z)>0$.\\
Z vlastnosti logaritmů platí:
\begin{gather*}
log(f(x,y,z))=log((1-e^x)(1-e^{2y})(1-e^{5z}))=log(1-e^x)+log(1-e^{2y})+log(1-e^{5z})
\end{gather*}
Funkce $1-e^x, 1-e^{2y}, 1-e^{5z}$ jsou konkávní, jelikož $e^x$ je konvexní $\implies$ $-e^x$ je konkávní a jsou posunuté akorát o konstantu, což nezmění konkávnost. $log$ je konkávní funkce a je neklesající. Z věty z přednášky tedy víme, že všechny sčítance jsou konkávní, tím pádem i součet ($log(f)$) těchto funkcí je konkávní. Tedy $f$ je log-konkávní.

\section{}
Pro $n=1$ je problém triviální. Pro $n=2$ jsme to dokázali v minulém úkolu. Nyní předpokládejme, že $f_n(x_1,\dots,x_n)=ln(exp(x_1)+\dots+exp(x_n))$ je konvexní a dokážeme, že $f_{n+1}=ln(exp(x_1)+\dots,exp(x_{n+1}))$ je konvexní funkce. Učiníme pozorování, že platí $f_{n+1}(x_1,\dots,x_{n+1})=f_2(f_n(x_1,\dots,x_n),x_{n+1})$. Pozorování platí díky tomu, že funkce $ln$ a $exp$ jsou navzájem inverzní. Takže:
\begin{gather*}
f_{n+1}(x_1,\dots,x_{n+1})=f_2(f_n(x_1,\dots,x_n),x_{n+1}) = ln(exp(f_n(x_1,\dots,x_n))+exp(x_{n+1}))=\\
=ln(exp(f_n(x_1,\dots,x_n))+exp(x_{n+1}))=ln(exp(ln(exp(x_1)+\dots+exp(x_n)))+exp(x_{n+1}))=\\
=ln(exp(x_1)+\dots+exp(x_n)+exp(x_{n+1}))
\end{gather*}
Nyní můžeme použít větu z přednášky o "skládání konvexních funkcí". $f_2$ je konvexní a neklesající v každé souřadnici, protože pokud zafixujeme $y \in \mathbb{R}$ a uvažujeme funkcí $\forall x\in \mathbb{R}: f_2(x,y)$ jako funkci jedné proměné, tak její derivace je $\forall x \in \mathbb{R}:\frac{exp(x)}{exp(x)+exp(y)}$. Tento zlomek je vždy kladný, jelikož $exp(x)$ je kladná funkce, takže $f_2$ je neklesající v souřadnici $x$. Analogicky to lze dokázat pro druhou souřadnici. Dále víme, že obě vnitřní funkce $f_n$ (z indukčního předpokladu) a $x_{n+1}$ (zřejmě) jsou konvexní. Z věty tedy víme, že i $f_{n+1}$ je konvexní.

\section{}

\end{document}