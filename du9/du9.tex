\documentclass[12pt, a4paper]{article}
\usepackage[margin=1in]{geometry}
\usepackage[utf8x]{inputenc}
\usepackage{indentfirst} %indentace prvního odstavce
\usepackage{mathtools}
\usepackage{amsfonts}
\usepackage{amsmath}
\usepackage{amssymb}
\usepackage{graphicx}
\usepackage{enumitem}
\usepackage{subfig}
\usepackage{float}
\usepackage[czech]{babel}
\usepackage{mathdots}
\usepackage{slashbox}

\begin{document}

\section{}
Ze zadání $p^* = \frac{1}{3}$. LP splňuje Slaterovo kritérium, takže máme \uv{strong duality} $p^*=d^*$. Chceme najít $(\lambda^*,\nu^*)$, tž. jsou to optimální řešení duálu. Definujeme Lagrangian:
\begin{gather*}
L(x,\lambda,\nu) = x_1^2+x_2^2+x_3^2-\lambda_1x_1-\lambda_2x_2-\lambda_3x_3+\nu x_1 + \nu x_2 + \nu x_3 - \nu\\
i = 1,2,3: \frac{\partial L}{\partial x_i} (x,\lambda,\nu)=2x_i-\lambda_i+\nu
\end{gather*}
Vidíme, že Lagrangian je konvexní funkce, tudíž jeho globální minumim bude v místě, kde je nulový gradient. Musí platit tedy: 
\begin{gather*}
i = 1,2,3: \frac{\partial L}{\partial x_i} = 0 \iff x_i = \frac{\lambda_i - \nu}{2}\\
\end{gather*}
Z KKT podmínky dále máme další požadavek na potenciální $\lambda^*$. Tím je complementary slackness, tudíž $i=1,2,3: \lambda_i^* (-x_i^*) = 0$, ale $x_i^* \neq 0 \implies \lambda_i^* = 0$. Výpočet se nyní zjednoduší, víme tedy $\lambda^* = (0,0,0)^T$. Dopočítáme $\nu$. Z rovnic výše a KKT vidíme, že 
\begin{gather*}
\nu^* = 0-2x_i^*= -2\frac{1}{3}=-\frac{2}{3}
\end{gather*}
Tedy dual opt. solution je $((0,0,0)^T, -\frac{2}{3})$. Po dosazení (pro kontrolu) nám výjde i, že $p^* = d^*$.
\section{}
\section{}
\section{}
P1 má tedy 8 možností (umístění lodi). Očíslujeme si je od 1 do 8 v pořadí, jak jsou v zadání. P2 má 7 možností, kam střelit. To je právě 7 políček na hrací ploše. Očíslujeme si je od 1 do 7 (zleva doprava, shora dolů). Po tomto očíslování, můžeme definovat payoff matici $A$:
\begin{gather*}
A = \begin{pmatrix}
-1 & 1 & 1 & 1 & -1 & 1 & 1 & 1\\
1 & -1 & 1 & 1 & -1 & -1 & 1 & 1\\
1 & 1 & -1 & 1 & 1 & -1 & -1 & 1\\
1 & 1 & 1 & 1 & 1 & 1 & -1 & 1\\
-1 & 1 & 1 & -1 & 1 & 1 & 1 & 1\\
1 & -1 & 1 & -1 & 1 & 1 & 1 & -1\\
1 & 1 & -1 & 1 & 1 & 1 & 1 & -1
\end{pmatrix}
\end{gather*}
Použijeme obdobný postup pro nalezení worst case opt. řešení pro P1 a P2. LP pro zjištění worst case opt. řešení pro P1 bude vypadat:
\begin{gather*}
maximize \ t\\
s.t. \ p \succeq 0\\
\sum_{i=1}^8 p_i = 1\\
i=1,\dots, 7: t \leq \tilde{a}_i^T p \text{, kde }\tilde{a}_i^T \text{ je }i. \text{ řádek matice } A
\end{gather*}
\end{document}
