\documentclass[12pt, a4paper]{article}
\usepackage[margin=1in]{geometry}
\usepackage[utf8x]{inputenc}
\usepackage{indentfirst} %indentace prvního odstavce
\usepackage{mathtools}
\usepackage{amsfonts}
\usepackage{amsmath}
\usepackage{amssymb}
\usepackage{graphicx}
\usepackage{enumitem}
\usepackage{subfig}
\usepackage{float}
\usepackage[czech]{babel}
\usepackage{mathdots}
\usepackage{slashbox}

\begin{document}

\section{}
\vspace{0.5\textheight}
\section{}
Máme $a \in \mathbb{R}^n$. Zvolme $x \in \mathbb{R}^n$ a dokážeme, že $x^T(aa^T)x \geq 0$ neboli $aa^T \in S^n_{+}$. Platí:
\[
x^T(aa^T)x = (a^Tx)^T(a^Tx) = ||a^Tx||
\]

Máme tedy euklidovskou normu vektoru pro kterou platí, že je vždy nezáporná, takže $\forall x,a \in \mathbb{R}^n: x^T(aa^T)x \geq 0 \iff ||a^Tx|| \geq 0$.

\section{}
\begin{enumerate}[label=\alph*)]
\item Množina je jednotková kružnice. Tato množina není konvexní, jelikož například bod $0,5 \cdot \begin{pmatrix}1 \\ 0\end{pmatrix} + 0,5 \cdot \begin{pmatrix}-1 \\ 0\end{pmatrix} = \begin{pmatrix}0 \\ 0\end{pmatrix}$ je konvexní kombinací bodů z této množiny, ale není na jednotkové kružnici. 
\item Z podmínky lze vyvodit, že každý bod v této množině musí mít kladné obě souřadnice neboli je to množina $\mathbb{R}^2_{++}$. Tato množina je konvexní, jelikož to je vlastně 1. kvadrant bez os. Množina není konvexní kužel, jelikož neobsahuje počátek.
\item Tato množina popisuje poloprostor, který obsahuje počátek. Je to opět konvexní množina. Obsahuje ale například bod $ \begin{pmatrix}\frac{2}{10} & \frac{-1}{10} & \frac{7}{10}\end{pmatrix}^T$, což je $\frac{1}{10}$ násobek bodu ze zadání. Tudíž leží na stejné přímce, která prochází počátkem, ale daný poloprostor jí neobsahuje celou. Takže to není kužel.
\item Množina není konvexní. Matice $\begin{pmatrix}1 & 0\\ 0 & 1\end{pmatrix}, \begin{pmatrix}-1 & 0\\ 0 & -1\end{pmatrix}$ jsou prvky dané množiny, ale jejich konvexní kombinace
\[
0,5\cdot \begin{pmatrix}1 & 0\\ 0 & 1\end{pmatrix} + 0,5 \cdot \begin{pmatrix}-1 & 0\\ 0 & -1\end{pmatrix} = \begin{pmatrix}0 & 0\\ 0 & 0\end{pmatrix}
\]
Má zřejmě nulový determinant.
\end{enumerate}

\section{}
\begin{enumerate}
\item Nechť $x,y \in \sqrt{2}A \implies \exists a_1,a_2 \in A: x=\sqrt{2}a_1, y = \sqrt{2}a_2$. Zvolme $\lambda \in [0,1]$ a dokážeme, že $\lambda x + (1-\lambda)y \in \sqrt{2}A \iff \exists c \in A : \lambda x + (1-\lambda)y = \sqrt{2}c$.
\[
\lambda x + (1-\lambda)y = \lambda \sqrt{2}a_1 + (1-\lambda)\sqrt{2}a_2 = \sqrt{2}(\lambda a_1 + (1-\lambda)a_2)
\]
$a_1,a_2 \in A$, kde $A$ je konvexní, takže položme $c \coloneqq \lambda a_1 + (1-\lambda)a_2 \in A$.
\item Nechť $x,y \in A+B \implies \exists a_1,a_2 \in A, b_1,b_2: x=a_1+b_1, y = a_2+b_2$. Zvolme $\lambda \in [0,1]$ a dokážeme, že $\lambda x + (1-\lambda)y \in A+B \iff \exists c_1 \in A, c_2 \in B: \lambda x + (1-\lambda)y = c_1 + c_2$.
\[
\lambda x + (1-\lambda)y= \lambda (a_1+b_1) + (1-\lambda)(a_2+b_2) = (\lambda a_1 + (1-\lambda)a_2) + (\lambda b_1 + (1-\lambda)b_2)
\]
$A,B$ jsou konvexní množiny, takže $\lambda a_1 + (1-\lambda)a_2 \in A, \lambda b_1 + (1-\lambda)b_2 \in B$. Položme $c_1 \coloneqq \lambda a_1 + (1-\lambda)a_2, c_2 \coloneqq \lambda b_1 + (1-\lambda)b_2$.

\section{}
Ze zadání plyne, že se nám nikdy nevyplatí kupovat $C$ za 80kč, když si jí můžeme vyrobit za 30kč ($3B = 1C)$. Označme $x_1$ počet kg suroviny $A$, které nakoupíme, $x_2$ počet kg suroviny $B$, které nakoupíme a $x_3$ počet kg suroviny $D$, které nám zbydou po reakcích. Definujme funkci $f(x_1,x_2,x_3) = 3x_1+10x_2+x_3$, která nám počítá celkovou cenu výroby. Chceme minimalizovat $f$.
\end{enumerate}

\end{document}