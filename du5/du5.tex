\documentclass[12pt, a4paper]{article}
\usepackage[margin=1in]{geometry}
\usepackage[utf8x]{inputenc}
\usepackage{indentfirst} %indentace prvního odstavce
\usepackage{mathtools}
\usepackage{amsfonts}
\usepackage{amsmath}
\usepackage{amssymb}
\usepackage{graphicx}
\usepackage{enumitem}
\usepackage{subfig}
\usepackage{float}
\usepackage[czech]{babel}
\usepackage{mathdots}
\usepackage{slashbox}

\begin{document}

\section{}
Z lineární algebry víme, že lineární zobrazení je jednoznačně určeno obrazy bázových prvků. Bází prostoru $S^n$ je například $\frac{n(n+1)}{2}$ symetrických matic, které mají právě na 2 místech (symetricky) hodnotu 1, pokud jsou tyto místa mimo diagonálu, nebo matice s jednou jedničkou na diagonále. Každou symetrickou matici dokážeme vyjádřit jako linární kombinaci těchto matic. Bází prostoru $S^2$ by například byly tyto matice:
\[
\begin{pmatrix}
1 & 0\\
0 & 0
\end{pmatrix}, 
\begin{pmatrix}
0 & 0\\
0 & 1
\end{pmatrix},
\begin{pmatrix}
0 & 1\\
1 & 0
\end{pmatrix}
\]
Položme $k \coloneqq \frac{n(n+1)}{2}$. Pro $i=1,\dots,k: B_i$ budou prvky báze popsané výše. Pro libovolné $X \in S^n$ tedy existují $i=1,\dots,k: a_i \in \mathbb{R}: X = \sum_{i=1}^k a_iB_i$. $f$ je lineární, takže $f(X)=f(\sum_{i=1}^k a_iB_i)=\sum_{i=1}^k a_if(B_i)$. Označme $i=1,\dots,k: b_i = f(B_i) \in \mathbb{R}$. Z toho, jak je zvolena báze, je zřejmé, že každý prvek $a_i$ je roven nějakému prvku $X_{i,j}=X_{j,i}$.


Víme, že platí $\forall C,X \in S^n: Tr(CX)= \sum_{i=1}^{n} \sum_{j=1}^{n} C_{i,j}X_{i,j}$. Protože $C,X$ jsou symetrické, tak sčítance mimo diagonálu ($i\neq j$) $C_{i,j}X_{i,j}$ a $C_{j,i}X_{j,i}$ jsou identické. Důsledkem je, že tuto sumu dokážeme napsat jako sumu $k=n+\frac{n^2-n}{2}$ prvků následovně:
\[
Tr(CX)=\sum_{i=1}^n(C_{i,i}X_{i,i})+2\sum_{i=1}^{n} \sum_{\substack{j=1}}^{i-1} C_{i,j}X_{i,j}
\]
Hledáme tedy matici $C \in S^n$. Porovnáme sčítance v sumách $f(X)=\sum_{i=1}^k a_ib_i$ a $Tr(CX)$ (mají obě $k$ sčítanců). Uvažujme například první sčítanec $C_{1,1}X_{1,1}$ a hledáme jeho odpovídající sčítanec $a_ib_i$. Jak bylo řečeno výše, tak víme, že $\exists i: a_i = X_{1,1}$ (jde jen o to jak si $a_i$ zaindexujeme). Musí platit $a_ib_i = X_{1,1}C_{1,1} \implies b_i = C_{1,1}$. Obdobně pro všechny diagonální prvky. 

Stručně řečeno, na diagonále $C$ jsou obrazy bázových prvků, které jsou diagonální matice. Pro zbylé prvky mimo diagonálu na indexech $i\neq j$ musí platit $\exists k: a_kb_k = 2C_{i,j}X_{i,j}$, kde $a_k = X_{i,j}=X_{j,i}$. Z toho plyne, že $C_{i,j}=C_{j,i}=\frac{1}{2}b_i$.


Výsledkem je, že na diagonále $C$ jsou obrazy bázových prvků, které jsou diagonální matice, a pro zbylé prvky je to polovina obrazu příslušného bázového prvku, který není diagonální.
\[
C = \begin{pmatrix}
f(B_{i_1}) & \frac{1}{2}f(B_{i_2}) & \dots \\
\frac{1}{2}f(B_{i_2}) & f(B_{i_3}) & \dots \\
\vdots & \ddots & \ddots \\
\end{pmatrix} \text{ pro příslušné $i_1,i_2,\dots$ (záleží na indexování bázových prvků)}
\]

\section{}
Definujme kužel $K \coloneqq \{(x_1,x_2,x_3,x_4,x_5,x_6)^T \in \mathbb{R}^6: \|(x_1,x_2)^T\| \leq x_3, \|(x_4,x_5\| \leq x_6\}$. Matice $F$ a vektory $g, c$ budou vypadat:
\[
F = -\begin{pmatrix}
1 & 0 & 0 \\
0 & 2 & 0 \\
0 & 0 & 1 \\
0 & 0 & 0 \\
0 & 1 & 0 \\
1 & 0 & 0
\end{pmatrix},
g = -\begin{pmatrix}
0 \\
2 \\
0 \\
1 \\
0 \\
1
\end{pmatrix},
c = \begin{pmatrix}
0\\
0\\
1
\end{pmatrix}
\]
Stačí nyní jen nahlédnout, že to odpovídá danému problému. Uvažujme vektor proměnných $y = (y_1, y_2, y_3)$. Podmínka $Fy+g \preceq_K 0$ odpovídá z definice tomu, že vektor
\[
v = \begin{pmatrix}
y_1 \\
2y_2 + 2\\
y_3\\
1\\
y_2\\
y_1+1
\end{pmatrix}
\]
je prvkem $K$. Což nastane právě tehdy, když $\|(y_1,2y_2+2)^T\|\leq y_3 \land \|(1,y_2)^T\| \leq y_1+1$. Což kopíruje podmínky zadání ($y_1 = x_1, y_2 = x_2, y_3 = t$). Minimalizujeme poslední složku vektoru $y$, tedy $y_3$ neboli $t$.

\section{}
Označíme si daných 14 předmětů čísly od 1 do 14 (1. je mapa a 14. je motorová pila). Proměnná $u_i$ bude značit \uv{efektivitu} i. předmětu v dz, $w_i$ jeho váhu v kg a $b_i \in [0,1]$ bude značit kolik daného přemětu si vybereme (1 odpovídá celému předmětu). Chceme maximalizovat $\sum_{i=1}^{14} b_i u_i = b^T u$ (celkovou efektivitu vybraných předmětů). Pokud problém vyjádříme jako LP problém tak to bude vypadat takto:
\begin{gather*}
\text{minimize } -b^T u\\
\text{s. t. } b \preceq 1\\
-b \preceq 0\\
b^T w \leq 10
\end{gather*}
První 2 podmínky zaručují, aby hodnoty $b_i$ byly v intervalu $[0,1]$. Poslední podmínka zaručuje, aby součet vah vybraných předmětů nepřekročil 10 kg.

Optimální řešení nám dává efektivitu $\approx$ 25.88. Ručně jsem nalezl řešení (spočítáním poměru \uv{váha výkon} u každého předmětu a vybíráním od nejlepšího) 1x mapa, 1x baterie, 1x kytara, 1x katana a 1x Jarník. Toto řešení má efektivutu 23. Samozřejmě optimální řešení nerelaxed problému by vyžadovalo vyzkoušení všech kombinací.

\section{}
\begin{enumerate}[label=\alph*)]
\item Uvažujme 2 matice $A,B \in \mathbb{R}^{5 \times 5}$. Vektor $Ax$ bude reprezentovat vyprodukované suroviny v roce 2019 aktivitami, vektor $Bx^{+}$ bude reprezentovat spotřebované suroviny v roce 2020. Matice $A$ a $B$ budou vypadat následovně:
\[
A = \begin{pmatrix}
-1 & 0 & -3 & 0 & 5\\
-1 & 0 & 0 & -1 & 2\\
0 & 0 & 10 & 0 & -2\\
3 & -1 & 0 & 1 & 0\\
-1 & 1 & -3 & 1 & -3
\end{pmatrix},
B = \begin{pmatrix}
1 & 0 & 3 & 0 & 0\\
1 & 0 & 0 & 1 & 0\\
0 & 0 & 0 & 0 & 2\\
0 & 1 & 0 & 0 & 0\\
1 & 0 & 3 & 0 & 3 
\end{pmatrix}
\]
Sloupce v matici $A$ reprezentují aktivity a řádky reprezentují suroviny. Například první sloupec odpovídá první aktivitě: spotřebujeme 1 kancelářské potřeby, 1 kafe a 1 štěstí a vyprodukujeme 3 slávy. Matice $B$ je tvořena obdobně, akorát reprezentuje pouze počet spotřebovaných surovin, takže hodnoty co byly v matici $A$ záporné (reprezentující spotřebování suroviny), tak jsou v $B$ kladné a původní kladné hodnoty z $A$ jsou v $B$ nulové, jelikož to, co se vyprodukuje v roce 2020 nás nezajímá. Zadefinujeme si funkci $f_0(x,x^{+}) = \text{max}_{i=1}^5 -\frac{x_i^{+}}{x_i}$. Nyní můžeme zformulovat GLFP:
\begin{gather*}
\text{minimize } f_0(x,x^{+})\\
\text{s. t. } Ax \preceq Bx^{+}\\
-x^{+} \preceq 0\\
-x \preceq 1 
\end{gather*}
Zřejmě platí $\text{max min} f = \text{min max} -f$. Proto jsme použili místo $g$ (ze zadání) funkci $f_0$. První nerovnost zaručí to, že spotřebované suroviny v 2020 nepřesáhnou vyprodukované v 2019. Podmínka na nezápornost $x^+$ je zřejmá (nemůžeme konat zápornou aktivitu) a poslední podmínka nám zaručí dobrou definovanost funkce $f_0$ ($x_i > 0$) a zároveň je jedno, jestli tam je $1$ nebo jiná kladná konstanta, jelikož nám záleží jen na poměrech mezi $x_i$.

\item Máme tedy $\alpha$ pevně dané. Chceme zjistit, jestli pro všechna $i=1,\dots,5$ platí $\frac{x_i^+}{x_i} \geq \alpha$. Jestli ano, tak zřejmě $g(x,x^+)\geq \alpha$ z definice. Podmínku se zlomkem můžeme ale ekvivalentně napsat jako $i=1,\dots,5: x_i^+ \geq \alpha x_i$. To dle našeho značení jde napsat vektorově jako $\alpha x \preceq x^+$. 

Můžeme nyní zformulovat lineární program Q (podobný GLFP), který bude mít optimální hodnotu 0, pokud existují dané vektory $x,x^+$ splňující $g(x,x^+)\geq \alpha$, a nebude mít žádné feasible solution pokud neexistují daná $x,x^+$.
\begin{gather*}
\text{minimize } 0\\
\text{s. t. } Ax \preceq Bx^{+}\\
-x^{+} \preceq 0\\
-x \preceq 1 \\
\alpha x \preceq x^+
\end{gather*}

\item email
\end{enumerate}
\end{document}
