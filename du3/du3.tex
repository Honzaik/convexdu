\documentclass[12pt, a4paper]{article}
\usepackage[margin=1in]{geometry}
\usepackage[utf8x]{inputenc}
\usepackage{indentfirst} %indentace prvního odstavce
\usepackage{mathtools}
\usepackage{amsfonts}
\usepackage{amsmath}
\usepackage{amssymb}
\usepackage{graphicx}
\usepackage{enumitem}
\usepackage{subfig}
\usepackage{float}
\usepackage[czech]{babel}
\usepackage{mathdots}
\usepackage{slashbox}

\begin{document}

\section{}
\begin{enumerate}[label=\alph*)]
 \item \[
f(x) = x^p, \forall x \in (0,\infty), p \geq 1
\]
\[
f'(x) = px^{p-1}, f''(x)=p(p-1)x^{p-2}
\]
$p \geq 1$, takže $p(p-1) \geq 0$ a zároveň $\forall x \in (0,\infty): x^{p-2} > 0  \implies \forall x \in (0,\infty): f''(x) \geq 0$. Interval $(0,\infty)$ je konvexní množina. Takže je funkce $f$ konvexní.
 \item Zvolme $x,y \in [0, \infty)$ a $\theta \in [0,1]$. Pokud $x = y = 0$, tak je konvexita zřejmá. Pokud $x,y \in (0,\infty)$, tak konvexita plyne z a). Zbývá tedy případ, kdy BÚNO $x \in (0,\infty), y=0$.
\[
g(\theta x + (1-\theta)y) = (\theta x + (1-\theta)y)^p = (\theta x)^p = \theta^p x^p \stackrel{\theta \in [0,1]}{\leq} \theta x^p = \theta g(x) + (1-\theta)g(y)
\]

$[0,\infty)$ je konvexní množina. Takže $g$ je konvexní funkce na $[0,\infty)$. 

\end{enumerate}


\section{}

\end{document}