\documentclass[10pt, a4paper]{article}
\usepackage[margin=1in]{geometry}
\usepackage[utf8x]{inputenc}
\usepackage{indentfirst} %indentace prvního odstavce
\usepackage{mathtools}
\usepackage{amsfonts}
\usepackage{amsmath}
\usepackage{amssymb}
\usepackage{graphicx}
\usepackage{enumitem}
\usepackage{subfig}
\usepackage{float}
\usepackage[czech]{babel}
\usepackage{mathdots}
\usepackage{slashbox}

\begin{document}

\section{}
\begin{enumerate}[label=\alph*)]
 \item \[
f(x) = x^p, \forall x \in (0,\infty), p \geq 1
\]
\[
f'(x) = px^{p-1}, f''(x)=p(p-1)x^{p-2}
\]
$p \geq 1$, takže $p(p-1) \geq 0$ a zároveň $\forall x \in (0,\infty): x^{p-2} > 0  \implies \forall x \in (0,\infty): f''(x) \geq 0$. Interval $(0,\infty)$ je konvexní množina. Takže je funkce $f$ konvexní.
 \item Zvolme $x,y \in [0, \infty)$ a $\theta \in [0,1]$. Pokud $x = y = 0$, tak je konvexita zřejmá. Pokud $x,y \in (0,\infty)$, tak konvexita plyne z a). Zbývá tedy případ, kdy BÚNO $x \in (0,\infty), y=0$.
\[
g(\theta x + (1-\theta)y) = (\theta x + (1-\theta)y)^p = (\theta x)^p = \theta^p x^p \stackrel{\theta \in [0,1]}{\leq} \theta x^p = \theta g(x) + (1-\theta)g(y)
\]

$[0,\infty)$ je konvexní množina. Takže $g$ je konvexní funkce na $[0,\infty)$. 

\end{enumerate}


\section{}
$\Rightarrow$: Zvolme $\begin{pmatrix} x_1 \\ t_1\end{pmatrix}, \begin{pmatrix} x_2 \\ t_2\end{pmatrix} \in \{\begin{pmatrix} x \\ t\end{pmatrix}: t \geq f(x), x \in dom(f), t \in \mathbb{R} \} \eqqcolon E$, $\theta \in [0,1]$ a dokážeme, že $\theta \begin{pmatrix} x_1 \\ t_1\end{pmatrix} + (1-\theta) \begin{pmatrix} x_2 \\ t_2\end{pmatrix} \in E$.
\begin{gather*}
\theta \begin{pmatrix} x_1 \\ t_1\end{pmatrix} + (1-\theta) \begin{pmatrix} x_2 \\ t_2\end{pmatrix} = \begin{pmatrix} \theta x_1 + (1-\theta) x_2 \\ \theta t_1 + (1-\theta) t_2\end{pmatrix}
\end{gather*}
Chceme dokázat, že 
\begin{enumerate}
\item $\theta x_1 + (1-\theta) x_2 \in dom(f)$
\item $\theta t_1 + (1-\theta) t_2 \geq f(\theta x_1 + (1-\theta) x_2)$
\end{enumerate}
1. plyne z toho, že $f$ je konvexní (tedy i $dom(f)$ je konvexní).\\
2. plyne z toho, že máme
\begin{gather*}
t_1 \geq f(x_1)\\
t_2 \geq f(x_2)
\end{gather*}
a konvexnosti $f$ viz
\[
f(\theta x_1 + (1-\theta) x_2) \leq \theta f(x_1) + (1-\theta)f(x_2) \leq \theta t_1 + (1-\theta) t_2
\]
$\Leftarrow$:
Zvolme $x_1, x_2 \in dom(f), \theta \in [0,1]$. Uvažujme body $\begin{pmatrix} x_1 \\ f(x_1)\end{pmatrix}, \begin{pmatrix} x_2 \\ f(x_2)\end{pmatrix}$. Tedy body jsou v množině $E$ (epigraf) z definice. Předpokládáme, že epigraf je konvexní množina, takže:
\begin{gather*}
\begin{pmatrix} \theta x_1 + (1-\theta) x_2 \\ \theta f(x_1) + (1-\theta) f(x_2)\end{pmatrix} \in E \implies\\
\theta x_1 + (1-\theta) x_2 \in dom(f)\\
\theta f(x_1) + (1-\theta) f(x_2) \geq f( x_1 + (1-\theta) x_2)
\end{gather*}
Takže $f$ je konvexní funkce.

\section{}
Konvexnost dokážeme pomocí hesiánu. Zřejmě $dom(f) = \mathbb{R}^2$ (konvexní množina).
\begin{gather*}
\frac{\partial^2 f}{\partial^2 x} =  \frac{\partial^2 f}{\partial^2 y} = \frac{e^{x+y}}{(e^x+e^y)^2}\\
\frac{\partial^2 f}{\partial x \partial y} = \frac{\partial^2 f}{\partial y \partial x} = -\frac{e^{x+y}}{(e^x+e^y)^2} \implies
H(f) = \frac{1}{(e^x+e^y)^2}
\begin{pmatrix}
e^{x+y} & -e^{x+y}\\
-e^{x+y} & e^{x+y}
\end{pmatrix}\\
\forall x,y \in \mathbb{R}: \frac{1}{(e^x+e^y)^2} \geq 0 \implies
|H(f)| \geq 0 \iff 
\begin{vmatrix}
e^{x+y} & -e^{x+y}\\
-e^{x+y} & e^{x+y}
\end{vmatrix} \geq 0 \implies\\
\begin{vmatrix}
e^{x+y} & -e^{x+y}\\
-e^{x+y} & e^{x+y}
\end{vmatrix} = 2(e^{x+y})^2 \implies \forall x,y \in \mathbb{R}: |H(f)| \geq 0
\end{gather*}
Zároveň $\forall x,y \in \mathbb{R}: e^{x+y} \geq 0$, takže $H(f)$ je pozitivně semidefinitní, takže $f$ je konvexní na $\mathbb{R}^2$.

\section{}
Pokud budeme uvažovat jeden den, tak produktivitu trpaslíků spočítáme jednoduše jako $\sum^{21}_{i=1} a_i x_i = d_i$, kde $a_i \in \{0,1\}$ značí, zda byl $i$. trpaslík přítomen, $x_i$ bude jeho produktivita v ten den a $d_i$ je vytěžené zlato v tento den. Pokud tedy máme informace jen o jednom dni, tak nám výjde, že každý přítomný trpaslík je stejně výkonný. Pokud tedy sestavíme soustavu rovnic pro všechny dny, tak máme soustavu rovnic:
\begin{gather*}
a_{1,1}x_1 + \dots + a_{1,21}x_{21} = d_1\\
\vdots\\
a_{30,1}x_1 + \dots + a_{30,21}x_{21} = d_{30}
\end{gather*}
kde $a_{i,j}$ značí přítomnost $j$. trpaslíka v $i$. den, jinak $x_i, d_i$ mají stejný význam. Soustavu tedy můžeme vyjádřit maticově a vyřešit pomocí metody nejmenších čtverců. Metoda nejmenších čtverců nám dá nejlepší aproximaci produktivity trpaslíků, kterou z dostupných dat můžeme určit. Produktivitou myslíme průměrný počet vytěženého zlata za den.

\section{}
\begin{enumerate}[label=\alph*)]
\item Chceme dokázat konkávnost $S$, takže stačí dokázat konvexnost $f \coloneqq -S$. $f: (0,1)^n \rightarrow \mathbb{R}$ na $dom(S)$. Zřejmě platí $dom(S) \subset (0,1)^n$. $f$ můžeme vyjádřit takto:
\begin{gather*}
f(x) = \sum^{n}_{i=1} g_i(x_i) \text{, kde } g_i(t): (0,1) \rightarrow \mathbb{R}, g_i(t) = tln(t)
\end{gather*}
$g_i$ je tedy funkce jedné proměnné. $\forall t \in (0,1): g''(t)=\frac{1}{t} \geq 0$. $g_i$ je tedy konvexní na $(0,1)$. Z přednášky víme, že součet konvexních funkcí je konvexní.\\
Nyní dokážeme konvexitu $dom(S)$. Zvolme $x,y \in dom(S) \subset (0,1)^n, \theta \in [0,1]$. Víme, že $\sum^n_{i=1} x_i = \sum^n_{i=1} y_i = 1$. Chceme dokázat, že $\sum^n_{i=1} (\theta x + (1-\theta)y)_i = 1$. To, že $\forall i: \theta x_i + (1-\theta) y_i \in (0,1)$ plyne ze zřejmé konvexnosti $(0,1)$, tedy $\theta x + (1-\theta)y \in (0,1)^n$. 
\begin{gather*}
\sum^n_{i=1} (\theta x + (1-\theta)y)_i = \sum^n_{i=1}(\theta x_i + (1-\theta)y_i)  = \sum^n_{i=1}(\theta x_i) + \sum^n_{i=1}((1-\theta)y_i) = \\
= \theta \sum^n_{i=1} x_i + (1-\theta)\sum^n_{i=1} y_i = \theta 1 + (1-\theta)1 = 1
\end{gather*}
Takže $dom(S)$ je konvexní $\implies f$ je konvexní funkce na $dom(S) \implies$ S je konkávní.
\item email
\end{enumerate}
\end{document}